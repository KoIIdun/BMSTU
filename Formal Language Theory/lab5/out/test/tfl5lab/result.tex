\documentclass[14pt,russian]{scrartcl}
\usepackage{geometry}
\geometry{a4paper,tmargin=2cm,bmargin=2.8cm,lmargin=3cm,rmargin=1cm}
\usepackage{tempora}
\usepackage{cmap}
\usepackage[T2A]{fontenc}
\usepackage[utf8]{inputenc}
\usepackage[english, main=russian]{babel}
\linespread{1.3}
\title{Отчет по лабораторной работе №5}
\author{Плаунов С.С\\Группа: ИУ9-51б\\Преподаватель: Непейвода А.Н}
\begin{document}
\maketitle
\section{Задача}
 Задача состояла в том чтобы сделать генератор \texttt{Latex}
 отчетов по лабораторным работам.
\section{Реализация}
 Реализую на языке \texttt{haskell.}
 Открытие листинга обозначено \{-BEGIN\_L \texttt{KEY}
 = \{...\} \texttt{NAME}
 = \{...\}-\}, закрытие \{-END\_L-\}. Парсинг файла lab\_report.bmstu игнорирует пробелы в не титульной части. Сначала экранируем символы: \textbackslash, \textnumero, \_, \$, \&, \%, \{, \}. После этого выделяем латиницу: \texttt{data.set(0),}
 \texttt{map.empty(),}
 \texttt{intercalate}
 Далее парсим системы \textsc{Haskell}
 Листинги оборачиваем по заданию, при этом допускается вложеные листинги (представленны на рисунках 1 и 2) При этом внутренние комментарии удаляются \begin{figure}[htb]
\footnotesize
this term has description\begin{verbatim}
parseBodyLR [] _ _ = []
parseBodyLR (tag:tags) (w:ws) result
 | length w == 0 = parseBodyLR (tag:tags) ws result
\end{verbatim}
\caption{desc}
\end{figure}
\begin{figure}[htb]
\footnotesize
\begin{verbatim}
parseBodyLR :: [[Char]] -> [[Char]] -> [[Char]] -> [[Char]]
parseBodyLR _ [] result = result
{- BEGIN_L KEY = {1} NAME = {desc} DESC = {this term has description}-}
parseBodyLR [] _ _ = []
parseBodyLR (tag:tags) (w:ws) result
 | length w == 0 = parseBodyLR (tag:tags) ws result
 | w /= tag = parseBodyLR (tag:tags) ws $ [(result !! 0) ++ " "++ w] ++ tail result
 | otherwise = parseBodyLR tags ws $ [""]++ result
\end{verbatim}
\caption{parseBodyLR}
\end{figure}
\section{Тестирование}
 Тестовый вариант создан из кода данной лабораторной, все пункты потестированны выше
\end{document}

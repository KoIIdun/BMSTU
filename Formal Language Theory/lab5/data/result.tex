\documentclass[14pt,russian]{scrartcl}
\usepackage{geometry}
\geometry{a4paper,tmargin=2cm,bmargin=2.8cm,lmargin=3cm,rmargin=1cm}
\usepackage{tempora}
\usepackage{cmap}
\usepackage[T2A]{fontenc}
\usepackage[utf8]{inputenc}
\usepackage[english, main=russian]{babel}
\linespread{1.3}
\title{Лабораторной работе }
\author{Петухова М.\\Группа: ИУ9-51б\\Преподаватель: Непейвода А.Н}
\begin{document}
\maketitle
\section{Задача}
 Выбрать язык (и среду) программирования, программы которого будут анализироваться генератором отчётов. Определить специальный тип комментария: псевдокод для генератора, — который будет извлекать для отчёта выделенные блоки программы. Комментарии этого специального типа должны быть двух видов: открывающий и закрывающий, причём каждый открывающий комментарий должен начинаться с ключей \texttt{KEY}
 = \{id листинга\} и \texttt{NAME}
 = \{имя листинга\}. Генератор отчётов по лабе принимает на вход проект на целевом ЯП, снабжённый псевдокодовыми комментариями, и дополнительный файл lab\_report.bmstu, содержащий данные для шапки и некоторые другие разделы отчёта, описанные в простой грамматике.
\section{Реализация}
 Комментарии в коде вида: \{- BEGIN\_COM \texttt{KEY}
 = \{...\} \texttt{NAME}
 = \{...\} -\}, закрытие \{- END\_COM-\}. Парсинг файла lab\_report.bmstu игнорирует пробелы в \texttt{TASK}
 \texttt{BODY}
 \texttt{TESTS.}
 Сначала экранируем пунктуацию. После этого выделяем код. Тег \textsc{Stack}
выделяется. Комментарии допускают описание.
\section{Тестирование}
 Пунктуация: \textbackslash, \textnumero, \_, \$, \&, \%, \{, \}. Код \texttt{import}
 Тег \textsc{Stack}
 \begin{figure}[htb]
\footnotesize
Descriptionexample\begin{verbatim}
commentCoords [] [] result = result
commentCoords [] stack result = []
commentCoords (x:xs) stack result =
 if tupcomments x == 0
 then commentCoords xs (x:stack) result
 else if length stack >= 1
      then commentCoords xs (tail stack) $ (head stack, x):result
      else []
\end{verbatim}
\caption{testcomment}
\end{figure}
\end{document}
